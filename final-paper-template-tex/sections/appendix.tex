\appendix

\section{Format Guidelines}

\subsection{Citations}
Citations within the text should include the author's last name and the year of publication, for example~\cite{example4}. Append lowercase letters to the year in cases of ambiguity. Treat multiple authors as in the following examples:~\cite{example1} or~\cite{example2} (for more than two authors) and \cite{example3} (for two authors). If the author portion of a citation is obvious, omit it, e.g., Nebel~\shortcite{example8}. Collapse multiple citations as follows:~\cite{example5,example6}.

\section{Reference Guidelines}
When citing references:
\begin{itemize}
\item Use a system for automatically generating bibliographic information from your database (e.g., BibTeX, Zotero, EndNote, Papers)
\item All ideas, fragments, figures, and data quoted from other work must be appropriately referenced
\item Literal quotations must be placed inside quotation marks and include exact page numbers
\item Paraphrases cannot be too close to the original wording
\item Every reference in the text must correspond to an item in the bibliography and vice versa
\end{itemize}

\subsection{Footnotes}
Place footnotes at the bottom of the page in a 9-point font. Refer to
them with superscript numbers.\footnote{This is how your footnotes
should appear.} Separate them from the text by a short
line.\footnote{Note the line separating these footnotes from the
text.} Avoid footnotes as much as possible; they interrupt the flow of
the text.

\subsection{Writing Style and Structure}
\subsubsection{Paragraph Construction}
Each paragraph should:
\begin{itemize}
\item Discuss one clear topic
\item Start with a clear topic sentence
\item Connect logically to surrounding paragraphs
\item Contribute to a clear line of argumentation from research question to conclusions
\end{itemize}

\subsubsection{Language and Style}
The paper should:
\begin{itemize}
\item Use proper English with correct grammar and spelling
\item Be free of lexical mistakes (use a grammar and spell checker before submission)
\item Be written objectively
\item Use engaging writing with varied sentence lengths and mixed active/passive voice
\item Avoid unnecessarily complicated or ambiguous sentences
\item Include critical review of existing scientific literature
\end{itemize}

\subsection{Formulas}
When typesetting formulas:
\begin{itemize}
\item Do not reduce formula sizes using {\tt small} or {\tt tiny} sizes
\item Split long formulas across multiple lines when they don't fit in a single line
\item Use {\tt resizebox} environment only for slightly long equations
\item Ensure equation numbers are in the same font and size as the main text (10pt)
\item Keep formula's main symbols no smaller than {\small small} text (9pt)
\end{itemize}

\subsection{Tables and Illustrations}
General guidelines for tables and illustrations:
\begin{itemize}
\item Place them where first discussed in the text
\item Float them to top (preferred) or bottom of the page
\item Include number and caption for each figure/table
\item Reference each figure/table at least once in the text
\item Include source references if copied from elsewhere
\item Make them independently interpretable through labeled axes, descriptive legends, etc.
\item Ensure they are understandable when printed in black and white
\end{itemize}

For tables specifically:
\begin{itemize}
\item Use the {\tt booktabs} package for better styling
\item Right-align numeric columns
\item Right-align corresponding headers
\item Use consistent precision for numbers
\item Avoid unnecessary repetition in content
\item Show units in column headers when possible
\end{itemize}

\subsection{Examples, Definitions, and Theorems}
When writing examples, definitions, theorems, and similar elements:
\begin{itemize}
\item Write them in their own paragraph
\item Separate by 2-5pt from surrounding paragraphs
\item Begin with item type in 10pt bold font followed by number
\item Optionally include title/summary in parentheses (non-bold)
\item Write main body in 10pt italics font
\item Use global numbering (e.g., Theorem 1, not Theorem 6.1)
\end{itemize}

\subsection{Algorithms and Listings}
For algorithms and listings:
\begin{itemize}
\item Treat them as special figures
\item Float them to top (preferred) or bottom of page
\item Place caption in header, left-justified between horizontal lines
\item Terminate algorithm body with horizontal line
\item Use appropriate packages ({\tt algorithm} and {\tt algorithmic} recommended)
\end{itemize}

\section{Additional Guidelines}

\subsection{Proofs}
When writing proofs:
\begin{itemize}
\item Write in separate paragraphs
\item Maintain 2-5pt separation from surrounding paragraphs
\item Start with ``Proof.'' in 10pt italics font
\item Use regular 10pt font for the proof content
\item End with unfilled square symbol (qed)
\item Use the \texttt{\textbackslash{proof}} environment in \LaTeX
\end{itemize}
 

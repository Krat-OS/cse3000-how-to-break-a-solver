\section{Related Work}

In the related work section of the paper, you should discuss the most important and relevant work that has been done in the field. This section should be structured in a way that helps the reader understand the context of your research and how it fits into the existing body of knowledge. You should also use this section to highlight the gaps in the current research and explain how your work addresses these gaps.

When writing the related work section, you should follow these guidelines:

\begin{itemize}
\item Start by providing a brief overview of the field and the key concepts that are relevant to your research.
\item Discuss the most important and influential work that has been done in the field. This should include both recent work and classic papers that have had a significant impact on the field.
\item Organize the related work into categories or themes that help the reader understand the different approaches that have been taken to address the research problem.
\item Highlight the gaps in the current research and explain how your work addresses these gaps. This should help the reader understand the novelty and significance of your research.
\item Be critical of the existing work and point out any limitations or weaknesses in the research that has been done so far. This will help you position your work in relation to the existing literature.
\item Use citations to support your claims and provide evidence for the points you are making. Make sure to cite all relevant work in the field and provide a comprehensive bibliography at the end of the paper.
\item Be clear and concise in your writing, and avoid unnecessary jargon or technical language that may be difficult for the reader to understand.
\end{itemize}

